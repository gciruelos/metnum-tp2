A continuación pasamos en limpio las conclusiones a las que llegamos con la realización del presente trabajo práctico.

Con respecto al método pagerank:
\begin{itemize}
	\item En lo concerniente a la convergerncia del método concluimos que:
		\begin{itemize}
			\item Para un mismo valor de $c$, la cantidad de iteraciones requeridas para distintos datasets varía poco, especialmente si $c$ es chico. Si $c$ es muy cercano a 1, entonces la cantidad de iteraciones es menor para grafos más disconexos.
			\item Entre más chico sea el valor de $c$, más rápido va a converger pagerank. Sin embargo, esto también produce que el resultado sea menos significativo, por lo cual es importante tener un buen criterio para la elección del $c$, por ejemplo, considerando la conectividaad del grafo.
			\item Al modificar la norma que usamos para determinar el criterio de parada, vemos que usar una norma $p$ disminuye la cantidad de iteraciones cuanto mayor es $p$, pero al mismo tiempo genera que el resultado sea menos preciso. De forma similar, un $p$ más chico produce el efecto contrario.
		\end{itemize}
	\item Sobre la performance del método, vimos que el paso más costoso del algoritmo es realizar el método de la potencia. Pese a que la cantidad de iteraciones para dos datasets distintos fuera prácticamente igual, el tiempo de cómputo efectivo era distinto pues el costo de realizar cada producto de matriz por vector dependía linealmente de la cantidad de coeficientes no nulos de la matriz (debido a la representación CRS utilizada para la misma). Por lo tanto, para matrices más esparsas (correspondientes a grafos más disconexos), se requiere menos tiempo de cómputo.
\end{itemize}

Con respecto al modelo GeM:
\begin{itemize}
	\item Vimos que el método propuesto por Govan et al. no es \emph{democrático}, en el sentido de que tiene mucha más relevancia sacarle un diferencia positiva a un equipo que está bien rankeado que a uno que no lo está. Esta es una clara diferencia  con el método tradicional de rankeo del fútbol, en la cual ganarle a cualquier equipo da la misma cantidad de puntos. Así mismo, las diferencias grandes de marcador son un factor clave pues favorecen más al vencedor, cosa que no se da con el método clásico.
	\item En la misma línea de lo anterior, vimos que el valor de $c$ afecta de forma directa la calidad del rankeo: si bien en términos generales puede no producir grandes cambios, el tomar un valor muy cercano a 1 genera que equipos de un bajo rendimiento general puedan aprovecharse demasiado de una victoria aislada a algún equipo bien posicionado. Mientras tanto, tomar un valor más cercano a 0 amortigua este efecto, dándole mayor importancia a la cantidad de partidos ganados. Particularmente, llegamos a la conclusión de que un valor para $c$ cercano a $0.35$ genera rankings sensatos (al menos para el caso de una liga de las características mencionadas en la sección (\ref{subsec:ligas})), logrando entonces un buen promedio entre la importancia de vencer a equipos importantes, así como marcar diferencias amplias de goles, y tener un \emph{ratio} razonablemente bueno de victorias/derrotas. Esto es especialmente importante en el caso del fútbol donde, a diferencia de otros deportes como el rugby o el tennis, no es raro que se produzcan resultados sorpresivos.  
	\item Por otro lado, encontramos que el no considerar los empates con el método GeM puede traer disparidades. Sin embargo, un análisis más profundo sobre este aspecto queda pendiente para trabajos posteriores, así como la realización de variaciones del método que manejen el caso de los empates de forma distinta. Algunas ideas en este sentido podrían ser poner en la posición $(i,j)$ y $(j,i)$ de la matriz de adyacencias la cantidad de goles por la que empataron los equipos $i$ y $j$, o bien, siempre poner 1. También queda en el tintero probar la versión generalizada del modelo GeM que se propone al final de \cite{Govan2008}, con la cual se podrían considerar otras estadísticas como posesión del balón, cantidad de tarjetas, etc.
\end{itemize}