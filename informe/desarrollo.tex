\subsection{Convenciones}

\subsection{Métodos numéricos usados}

Como dijimos en la introducción, nuestro objetivo será, dada una matriz de transición $P$, encontrarle un autovector de autovalor asociado igual a 1 a su transpuesta. (Usamos la transpuesta por comodidad notacional).

\[P^t x = x\].


\subsubsection{Método de la potencia}
El método de la potencia, dada una matriz A, produce un autovalor $\lambda$ y un autovector asociado a $\lambda$, $v$ no nulo. El método es iterativo, y se puede encontrar una mejor explicación sobre él en \cite[Cap. 5.8.1]{dahlquist}. 

Consiste en tomar un $x^{(0)}$ inicial, y luego construir una sucesión $\{x^{(k)}\}$ de la siguiente manera:

\[x^{(i)} = \frac{A x^{(i-1)}}{||A x^{(i-1)}||}\]

Y entonces, bajo ciertas condiciones, si se toma $k$ lo suficientemente grande, $x^{(k)} \to \overline{x}$, tal que $A\overline{x} = \lambda \overline{x}$, $\lambda$ el autovalor de mayor módulo. Por ello establecemos como criterio de parada que la diferencia entre el vector generado en una iteración y su anterior sea lo suficientemente chica.

Como probamos en \ref{subsub:prop1}, el autovalor de máximo módulo en este caso es 1, aunque puede pasar que $\lambda = -1$ también sea un autovalor. Sin embargo, comenzando con $x^{(0)} = (\frac1n,..., \frac1n)$ inicial, nos aseguramos de que las entradas sean siempre positivas, consiguiendo así un autovector asociado a autovalor $\lambda = 1$.

\subsubsection{PageRank}
PageRank más que un m\'etodo de cómputo es un m\'etodo de modelado. Dado un grafo cuyos nodos representan páginas webs y sus aristas representan links entre las páginas web, nos permitirá modelar un navegante aleatorio utilizando una cadena de Markov. 
Los detalles de la construcción de la cadena y la matriz asociada pueden encontrarse en \cite{Brin1998}.

Proveeremos una breve explicación de como se arma la matriz de transición utilizando un vector fila de la matriz $P$. 
$P_i$ es la $i$-esima fila de la matriz, y su entrada $j$-ésima nos dice la probabilidad que habrá de ir de la página web $i$ a la $j$. A priori una buena aproximación sería

\[ P_{ij} = \begin{cases} 
      \frac{1}{n_i} & \text{si hay un link de $i$ a $j$} \\
       0 & \text{si no}
   \end{cases}
\] 

Donde $n_i$ es la cantidad de links salientes de la página $i$.
El primer problema que se evidencia es que, en general, esta matriz no es de transición, porque si una página web no tiene links salientes, la matriz va a tener toda una fila de ceros. Por eso, en este caso, se agrega una fila que vale toda $(\frac1n, ..., \frac1n)$.

Luego, se introduce el concepto de teletransportación. La idea es que, con una cierta probabilidad $1-c$, el navegante aleatorio puede saltar a cualquier página de toda la red sin importar en cual esté actualmente. Todo esto, nuevamente, esta correctamente explicado en \cite{Brin1998} y \cite{Kamvar2003}.

Luego, puede utilizarse el algoritmo descripto anteriormente para encontrar un autovector de autovalor asociado igual a 1.

En este trabajo en particular, utilizaremos una versión mejorada del m\'etodo de la potencia, adaptada para este problema en particular, propuesta por \cite{Kamvar2003}. Este consiste en separar el único paso del método de la potencia en 3 pasos separados, de tal manera de acelerar el cómputo, aprovechándonos de que la matriz de transición (sin agregarle el factor de teletransportación) es esparsa.

De esta manera, se pueden obtener importantes ganancias en lo que respecta a la performance.

\subsubsection{Método GeM}

El método GeM, propuesto en \cite{Govan2008}, tiene como objetivo adaptar el algoritmo de PageRank para ligas deportivas. La idea es simple, al igual que en el algoritmo original de PageRank, la idea es armar una cadena de Markov y modelar un navegante aleatorio.

En este modelo, se representa una temporada (o una fecha, o un periodo de tiempo cualquiera) como un grafo dirigido y pesado, al igual que en el modelo de PageRank. Sin embargo, en este caso, los pesos de la primera matriz no valen 0 o 1, sino que son el valor absoluto de los puntajes de cada partido.

De esta manera, si el equipo $i$ perdió contra el equipo $j$ por $p$ puntos, en la primera matriz $A$, valdra que $A_{ij} = p$. 

Luego, al igual que en PageRank, las filas de esta matriz que valgan 0 (eso significa que el equipo está invicto hasta el momento) serán completadas y además se agregará el factor de teletransportación, haciendo que todas las entradas de la matriz $P$ sean distintas de 0.

Al igual que antes, nuestro objetivo es encontrar un autovector de autovalor 1 para $P^t$, y para ello utilizaremos el método de la potencia común y corriente.

Es importante notar que este m\'etodo es, a diferencia de el anterior, un m\'etodo que nos indica \emph{cómo} modelar, mientras que el m\'etodo propuesto en \cite{Kamvar2003} lo que hace es tomar un modelo conocido e intentar mejorar su velocidad de cómputo.

\subsubsection{Otros m\'etodos}

Además de los m\'etodos mencionados enteriormente, para cada problema (rankeo de páginas web y de ligas deportivas), utilizaremos un m\'etodo alternativo. 

En el caso de las páginas webs será el conocido como IN-DEGREE, que rankea mejor a aquellas páginas que son mas linkeadas y peor a las que son menos linkeadas. En la parte de experimentación se comparará estos m\'etodos con la requerida profundidad.

En el caso de los rankings de ligas deportivas, utilizaremos el m\'etodo standard de rankeo de cada deporte. En el caso del fútbol, este consiste en caso de empate un punto a cada equipo, y en otro caso darle 3 puntos al equipo ganador y 0 al perdedor.


\subsection{Estructuración del código}
Para el modelado del problema diseñamos tres módulos: Matriz, MatrizDep y Problema. 


\subsubsection{Matriz}
El módulo Matriz es el que se usará para representar a las matrices de conectividad de redes de páginas web.
\paragraph{Representación interna}

Como la matriz de conectividad es en general esparsa (dado que cada página web se conecta en proporción con muy pocas páginas), es conveniente utilizar una representación que aproveche esto. Para eso, vamos a usar una representación conocida como Compressed Row Storage (CRS). Para más información sobre este formato puede consultarse \cite{CRS}.

Elegimos este formato porque será especialmente cómodo a la hora de hacer el producto iterativo del método de la potencia, dado que si queremos hacer $P^tx$, es conveniente poder acceder a $P^t$ por filas fácilmente. 

Además, nos guardamos la cantidad de links que entran y salen de cada nodo. El primer dato será util para calcular la métrica IN-DEGREE, y el segundo dato será util para saber cuánto valdra $P^t_{ij}$, dado que es $\frac{1}{n_j}$, donde $n_j$ es la cantidad de nodos salientes del nodo $j$.

\paragraph{Interfaz}
La interfaz de Matriz provee las siguientes operaciones:\footnote{Cuando se escribe la aridad de la función la misma puede no coincidir con la notación usada en C++. Esto está bien pues lo único que se busca aquí es dar una orientación de lo que hace cada función y no código preciso.}

\begin{itemize}
    \item Matriz(\textbf{ifstream\&} in): constructor de la matriz. Recibe un archivo abierto del cual parseará todos los datos que necesita, en el formato adecuado (SNAP).

    \item \textbf{vector<double>} multiplicar(\textbf{vector<double>} x): El propósito de esta función es autoexplicativo. Devuelve el resultado del producto $Ax$, donde $A$ es la matriz de la clase. El algoritmo que utilizamos es el standard para multiplicar por matrices representadas en CRS, y además, como dijimos anteriormente, dividimos cada entrada por la cantidad de nodos salientes del nodo que corresponda.
\end{itemize}

\subsubsection{MatrizDep}
El módulo MatrizDep es el que se usará para representar a las matrices de conectividad de ligas deportivas.

\paragraph{Representación interna}

Como la matriz de conectividad en este caso, a diferencia del anterior, no suele ser esparsa (de hecho, en el caso general podría aproximarse bastante al grafo completo), entonces no fue necesario utilizar ninguna estructura compleja, y utilizamos simplemente la clásica representación de vector de vectores.

Además fue necesario utilizar otra estructura para almacenar los puntajes de acuerdo a otros métodos de rankeo (por ejemplo, el standard) de la liga deportiva correspondiente.

\paragraph{Interfaz}

La interfaz de MatrizDep provee las siguientes operaciones:
\begin{itemize}
    \item MatrizDep(\textbf{ifstream\&} in): constructor de la matriz. Recibe un archivo abierto del cual parseará todos los datos que necesita, en el formato adecuado (el explicitado en el tp).

    \item \textbf{vector<double>} multiplicar(\textbf{vector<double>} x): autoexplicativa. Devuelve el resultado del producto $Ax$, donde $A$ es la matriz de la clase. El algoritmo que utilizamos es el común y corriente, el mismo que se utiliza al hacer cuentas a mano con matrices.
\end{itemize}




\subsubsection{Problema}
Problema es un módulo que engloba todo lo relacionado al modelado y a la resolución de cada caso particular.


Adicionalmente, tenemos algunas funciones auxiliares: 

\paragraph{Interfaz}

La interfaz de Problema provee funciones utilizadas tanto para el modelado de rankings de páginas web como para el modelado de rankings en ligas deportivas.

Las siguientes operaciones en lo que concierne a la resolución de problemas relacionados con el modelado de páginas web:\footnote{Cuando se escribe la aridad de la función la misma puede no coincidir con la notación usada en C++. Esto está bien pues lo único que se busca aquí es dar una orientación de lo que hace cada función y no código preciso.}

\begin{itemize}
    \item  \textbf{vector<double>} pagerank(\textbf{Matriz} p\_trans, \textbf{double} c, \textbf{double} tolerancia) :
        Se ocupa de aplicar el algoritmo de PageRank sobre la matriz \texttt{p\_trans}, utilizando el algoritmo descripto por \cite[Algoritmo 1]{Kamvar2003}. Además, se testea luego de cada iteración si la diferencia (en norma 1\footnote{$||x||_1 = \sum_i |x_i|$}) es menor que la tolerancia. En ese caso, se termina de iterar y se considera que el vector resultante de la iteración actual es el resultado. 
          
    \item \textbf{vector<uint>} indeg(\textbf{Matriz} p\_trans):  Devuelve el resultado de aplicarle el m\'etodo de rankeo indegree a la red del problema.

A continuación siguen los m\'etodos utilizados para el modelado de rankings de ligas deportivas

  \item \textbf{vector<double>} metodopot(\textbf{MatrizDep} p, \textbf{double} tolerancia):
      M\'etodo de la potencia \emph{vanilla}. Es el que se utilizará para resolver el problema de las ligas deportivas. Obtiene una sucesión de vectores, y cuando su diferencia de norma 1 es menor que la tolerancia pasada como parámetro se termina de iterar.   

\end{itemize}

\subsection{Experimentación}

