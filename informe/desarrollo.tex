\subsection{Convenciones}

\subsection{Métodos numéricos usados}

Como dijimos en la introducción, nuestro objetivo será, dada una matriz de transición $P$, encontrarle un autovector de autovalor asociado igual a 1 a su transpuesta. (Usamos la transpuesta por comodidad notacional).

\[P^t x = x\].

\subsubsection{Método de la potencia}
El método de la potencia, dada una matriz A, produce un autovalor $\lambda$ y un autovector asociado a $\lambda$, $v$ no nulo. El método es iterativo, y se puede encontrar una mejor explicación sobre él en \cite[Cap. 5.8.1]{dahlquist}. 

El método consiste en tomar un $x^{(0)}$ inicial, y luego construir una sucesión $\{x^{(k)}\}$ de la siguiente manera:

\[x^{(i)} = \frac{A x^{(i-1)}}{||A x^{(i-1)}||}\]

Y entonces, bajo ciertas condiciones, si se toma $k$ lo suficientemente grande, $x^{(k)} \to \overline{x}$, tal que $A\overline{x} = \lambda \overline{x}$, $\lambda$ el autovalor de mayor módulo. Por ello establecemos como criterio de parada que la diferencia entre el vector generado en una iteración y su anterior sea lo suficientemente chica.

Como probamos en \ref{subsub:prop1}, el autovalor de máximo módulo en este caso es 1, pero puede pasar también que $\lambda = -1$ también sea un autovalor, pero comenzando con $x^{(0)} = (\frac1n,..., \frac1n)$ inicial, nos aseguramos de que las entradas sean siempre positivas, consiguiendo así un autovector asociado a autovalor $\lambda = 1$.

\subsubsection{PageRank}
PageRank será un método, que, dado un grafo cuyos nodos representan páginas webs y sus aristas representan links entre las páginas web, nos permitirá modelar un navegante aleatorio utilizando una cadena de Markov. 
Los detalles de la construcción de la cadena y la matriz asociada pueden encontrarse en \cite{Brin1998}.

Proveeremos una breve explicación de como se arma la matriz de transición utilizando un vector fila de la matriz $P$. 
$P_i$ es la $i$-esima fila de la matriz, y su entrada $j$-ésima nos dice la probabilidad que habrá de ir de la página web $i$ a la $j$. A priori una buena aproximación sería

\[ P_{ij} = \begin{cases} 
      \frac{1}{n_i} & \text{si hay un link de $i$ a $j$} \\
       0 & \text{si no}
   \end{cases}
\] 

Donde $n_i$ es la cantidad de links salientes de la página $i$.
El primer problema, obvio, es que en general, esta matriz no es de transición, porque si una página web no tiene links salientes, la matriz va a tener toda una fila de ceros. Por eso, en este caso, se agrega una fila que vale toda $(\frac1n, ..., \frac1n)$.

Luego, se introduce el concepto de teletransportación. La idea es que, con una cierta probabilidad $1-c$, el navegante aleatorio puede saltar a cualquier página de toda la red sin importar en cual esté actualmente. Todo esto, nuevamente, esta correctamente explicado en \cite{Brin1998} y \cite{Kamvar2003}.

En este trabajo en particular, utilizaremos una versión mejorada del algoritmo, propuesta por \cite{Kamvar2003}. Este consiste en separar el único paso del método de la potencia en 3 pasos separados, de tal manera de acelerar el cómputo, aprovechandonos de que la matriz de transición (sin agregarle el factor de teletransportación) es esparsa.

\subsubsection{Método GeM}

El método GeM, propuesto en \cite{Govan2008}, tiene como objetivo adaptar el algoritmo de PageRank para ligas deportivas. La idea es simple, al igual que en algoritmo original de PageRank, la idea es armar una cadena de Markov y modelar un navegante aleatorio.

En este modelo, se representa una temporada (o una fecha, o un periodo de tiempo cualquiera) como un grafo dirigido y pesado, al igual que en el modelo de PageRank. Sin embargo, en este caso, los pesos de la primera matriz no valen 0 o 1, si no que toman el valor del valor absoluto de los puntajes de cada partido.

De esta manera, si el equipo $i$ perdió contra el equipo $j$ por $p$ puntos, en la primera matriz $A$, valdra que $A_{ij} = p$. 

Luego, al igual que en PageRank, las filas de esta matriz que valgan 0 (eso significa que el equipo está invicto hasta el momento) serán completadas y además se agregará el factor de teletransportación, haciendo que todas las entradas de la matriz $P$ sean distintas de 0.

Al igual que antes, nuestro objetivo es encontrar un autovector de autovalor 1 para $P^t$, y para ello utilizaremos el método de la potencia común y corriente.

\subsection{Estructuración del código}

\subsection{Experimentación}

