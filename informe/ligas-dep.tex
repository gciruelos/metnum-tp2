El conjunto de datos que consideramos para esta sección es el correspondiente al actual torneo de primera división del fútbol argentino. Dicha liga cuenta con las siguientes características relevantes al problema que queremos tratar:
\begin{enumerate}
	\item Hay un total de 30 equipos que juegan todos una vez entre sí. Además, cada equipo juega un partido adicional con su clásico rival.
	\item Los empates son un resultado posible, e incluso muy frecuente. En nuestra implementación de GeM, la posición que tomamos frente a esta situación (que no está contemplada en \cite{Govan2008}) es la de no hacer nada: es decir, si los equipos $i$ y $j$ empataron (o más generalmente, la diferencia absoluta de goles entre todos los partidos que jugaron es 0), entonces en las posiciones $(i,j)$ y $(j,i)$ de nuestra matriz de adyacencia ponemos 0. Consideramos que esta medida es más que razonable, pues en definitiva lo que busca el algoritmo GeM es establecer una \emph{relación de fuerza} entre los equipos a partir de la diferencia de goles que se sacan, la cual, en el caso concreto del empate, es nula. 
	\item El sistema de puntuación estándar es: 3 puntos por partido ganado, 1 por empate y 0 por derrota.
	\item Al momento de realizarse el presente trabajo, sólo se ha jugado hasta la fecha 26 del torneo. Por lo tanto sólo analizaremos los datos hasta la misma. 
\end{enumerate} 

A diferencia de lo que sucedía en la sección (\ref{subsec:pagerank}), aquí no nos interesa realizar un análisis sobre la performance o la convergencia del método, sino más bien un análisis sobre la \emph{calidad} (en algún sentido) de los rankings generados por el mismo, contrastándolos con los reales. Por lo tanto las hipótesis irán en esa dirección. A continuación las enlistamos.

\begin{itemize}
	\item No todos los goles tendrán el mismo peso. Esto es lógico si consideramos la naturaleza del método: si el equipo A le gana al equipo B, que está invicto hasta el momento, por X goles, esto impactará mucho más en el rankeo de A que si le hubiera ganado por X goles a C, un equipo que pierde usualmente. Esto ya plantea una diferencia importante con el método de puntuación estándar, donde en cualquiera de los dos casos A mejoraría su puntaje de la misma manera.

	\item Algo interesante para analizar es la influencia de los empates en los dos tipos de rankeos. Como ya se dijo, un empate en nuestra implementación de GeM no altera la relación de fuerzas entre los equipos. De hecho, a los fines prácticos, un empate es equivalente a que ambos equipos nunca hayan jugado entre sí (en nuestro modelo), por lo que es claro que la matriz de GeM antes o después de un empate es la misma. Sin embargo, con el método estándar los empates sí que modifican el ranking. No alteran la posición relativa de los equipos que empataron, pero claramente puede mejorar la posición de los dos equipos frente al resto (ya que ambos sumaron un punto). Ante esta situación, podemos suponer que los empates van a tener un efecto no despreciable en los resultados reales, siendo una potencial fuente de diferencias con la versión GeM.

	\item Para valores de \emph{c} \footnote{Recordar que 1-\emph{c} era la probabilidad de teletransportación, o en este caso más concretamente, la probabilidad de que un equipo cualquiera pueda perder contra cualquier otro.} muy chicos los puntajes de los equipos deberían tender a homogeneizarse, generando rankeos poco deseables considerando la \emph{performance} de los equipos. Como contrapartida, para valores de \emph{c} muy altos, el rankeo debería ser más justo.
\end{itemize}


\subsubsection{Análisis cualitativo de los resultados}

\begin{table}[H]
	\center
	\begin{flushright}
	\begin{tabular}{| m{0.405\textwidth} || m{0.385\textwidth} |}
		\rowcolor{LightCyan}
		\hline Método GeM & Método Estándar \\ \hline
	\end{tabular}

	\begin{tabular}{| c | c | c || c | c |}
	  	\hline
	  	\rowcolor{LightCyan}
	  	Posición & Equipo & Puntaje & Equipo & Puntaje \\ \hline \hline
		1 & Boca Juniors & 0.0859387 & Boca Juniors & 58 \\ \hline
		2 & \cellcolor{green!20} Aldosivi & 0.0653494 & San Lorenzo & 54 \\ \hline
		3 & River Plate & 0.063251  & Rosario Central & 52\\ \hline
		4 & San Lorenzo & 0.0624152 & Racing Club & 49\\ \hline
		5 & Rosario Central & 0.048369 & River Plate & 48\\ \hline
		6 & Racing Club & 0.0480862 & Independiente & 45\\ \hline
		7 & San Martín (SJ) & 0.0439307 & Banfield & 43\\ \hline
		8 & Quilmes & 0.0421091 & Belgrano & 43\\ \hline
		9 & Newell's Old Boys & 0.038184 & Tigre & 42\\ \hline
		10 & Vélez Sarsfield & 0.0374806 & Estudiantes (LP) & 42\\ \hline
		11 & Independiente & 0.0363642 & Lanús & 41\\ \hline
		12 & Belgrano & 0.0362427 & Quilmes & 39\\ \hline
		13 & Gimnasia y Esgrima (LP) & 0.0334854 & Unión & 38\\ \hline
		14 & Banfield & 0.0308403 & Gimnasia y Esgrima (LP) & 37\\ \hline
		15 & Estudiantes (LP) & 0.029259 & Newell's Old Boys & 33\\ \hline
		16 & Unión & 0.0287605 & San Martín (SJ) & 32\\ \hline
		17 & Tigre & 0.0272169 & Sarmiento & 30\\ \hline
		18 & Sarmiento & 0.0258276 & \cellcolor{green!20} Aldosivi & 30\\ \hline
		19 & Lanús & 0.0251797 & Temperley & 29\\ \hline
		20 & Huracán & 0.0247069 & Argentinos Juniors & 29\\ \hline
		21 & Defensa y Justicia & 0.0239225 & Olimpo & 29\\ \hline
		22 & Olimpo & 0.0224303 & Defensa y Justicia & 27 \\ \hline
		23 & Arsenal & 0.0208047 & Huracán & 26 \\ \hline
		24 & Godoy Cruz & 0.0174517 & Vélez Sarsfield & 26 \\ \hline
		25 & Temperley & 0.0158447 & Godoy Cruz & 25 \\ \hline
		26 & Crucero del Norte & 0.0156778 & Colón & 24 \\ \hline
		27 & Argentinos Juniors & 0.0153485 & Arsenal & 23 \\ \hline
		28 & Nueva Chicago & 0.0142273 & Atlético de Rafaela & 22 \\ \hline
		29 & Atlético de Rafaela & 0.0112011 & Nueva Chicago & 17 \\ \hline
		30 & Colón & 0.010094 & Crucero del Norte & 14 \\ \hline
	\end{tabular}
	\end{flushright}
	\caption{\footnotesize Ranking correspondiente a la fecha 26 usando el método GeM, para un valor de $c = 0.85$, y el método estándar.}
	\label{tab:fecha26}
\end{table}

Al observar la tabla \ref{tab:fecha26}, notamos como primera gran sorpresa que Aldosivi (8 ganados-12 perdidos) aparece segundo en nuestro ranking GeM, siendo que en la realidad ocupa la posición 18. Este resultado, \emph{a priori} extraño, parece cobrar sentido si consideramos lo sucedido en la fecha 13 del torneo: Aldosivi goleó por 3-0 a Boca, el actual puntero según nuestro ranking (y también según el ranking oficial). En efecto, si vemos la tabla \ref{tab:fecha12-13} podemos apreciar como la contundente derrota de Boca (que hasta ese momento estaba primero) permitió catapultar a Aldosivi del cuarto al primer puesto, con una diferencia respecto del segundo particularmente amplia. 

Vale destacar, que en la realidad, este resultado tuvo un impacto significativamente menor: Boca siguió estando primero, con la pequeña diferencia de que fue alcanzado por San Lorenzo, mientras que Aldosivi se mantuvo lejos de la punta, en el puesto 8.

En los sucesivos partidos, Aldosivi sufrió una larga lista de derrotas, sin embargo, como vimos en la tabla \ref{tab:fecha26} esto apenas si perjudicó su posición. Podemos decir que esto es producto del \emph{batacazo} que dió frente a Boca, que mantuvo durante casi la totalidad del torneo su hegemonía por sobre los otros equipos. Dicho de otra forma, la suerte de Aldosivi quedó en algún punto supeditada a la de Boca, pues entre mejor le fuera a este último, más significativos serían los 3 goles que le había hecho para su propio puntaje. Esto sucede a tal punto que, a la fecha siguiente, Boca pierde 2-0 con Velez pero en lugar de perjudicarlo genera que, posiblemente en combinación con el hecho de que Aldosivi también perdiera por una diferencia de un gol, y de que River (quien había perdido el superclásico) ganara por 2, vuelva a estar puntero, desplazando a Aldosivi al tercer puesto. Es decir que lo importante para rescatar de este úlimo hecho, es que que al finalizar la fecha 14 los 3 goles de Aldosivi ante Boca perdían un poco de peso, pues ahora, en total, este tenía 2 goles más en su contra.

Luego, por todo lo dicho hasta aquí, creemos que nuestra primer hipótesis, concerniente a la distinta importancia que se le asigna a los goles según la performance del equipo que los recibe, queda verificada.  

\begin{table}[H]
	\center
	%\begin{flushright}
	\hspace{0.105\textwidth}
	\begin{tabular}{| m{0.297\textwidth} || m{0.2975\textwidth} |}
		\rowcolor{LightCyan}
		\hline Fecha 12 & Fecha 13 \\ \hline
	\end{tabular}
	

	\begin{tabular}{| c | c | c || c | c |}
	  	\hline
	  	\rowcolor{LightCyan}
	  	Posición & Equipo & Puntaje & Equipo & Puntaje \\ \hline \hline
		1 & \cellcolor{blue!20} Boca Juniors & 0.108944 &  \cellcolor{green!20} Aldosivi & 0.117217 \\ \hline
		2 & River Plate & 0.0815019 & \cellcolor{blue!20} Boca Juniors & 0.0892615 \\ \hline
		3 & Rosario Central & 0.0735106 & River Plate & 0.0657072 \\ \hline
		4 & \cellcolor{green!20} Aldosivi & 0.0546487 & Rosario Central & 0.0608375 \\ \hline
		5 & San Lorenzo & 0.0511854 & Banfield & 0.0546665 \\ \hline
		% 6 & Racing Club & 0.046461 & San Lorenzo & 0.0497559 \\ \hline
		% 7 & Belgrano & 0.0464054 & Gimnasia y Esgrima (LP) & 0.0478387 \\ \hline
		% 8 & Banfield & 0.0440866 & Vélez Sarsfield & 0.0426006 \\ \hline
		% 9 & San Martín (SJ) & 0.0433587 & Racing Club & 0.0418281\\ \hline
		% 10 & Newell's Old Boys & 0.0353829 & Newell's Old Boys & 0.036451 \\ \hline
		% 11 & Lanús & 0.0320765 & Belgrano & 0.0356502 \\ \hline
		% 12 & Gimnasia y Esgrima (LP) & 0.029684 & San Martín (SJ) & 0.0350081 \\ \hline
		% 13 & Sarmiento & 0.0263866 & Lanús & 0.0346346 \\ \hline
		% 14 & Arsenal & 0.0252073 & Estudiantes (LP) & 0.025968 \\ \hline
		% 15 & Huracán & 0.0241481 & Argentinos Juniors & 0.0241652 \\ \hline
		% 16 & Estudiantes (LP) & 0.024007 & Huracán & 0.0215508 \\ \hline
		% 17 & Tigre & 0.023322 & Godoy Cruz & 0.0211769 \\ \hline
		% 18 & Vélez Sarsfield & 0.0229058 & Sarmiento & 0.0210192 \\ \hline
		% 19 & Independiente & 0.0224838 & Arsenal & 0.0200022 \\ \hline
		% 20 & Argentinos Juniors & 0.0221471 & Unión & 0.0199389 \\ \hline
		% 21 & Temperley & 0.0212702 & Defensa y Justicia & 0.019309 \\ \hline
		% 22 & Unión & 0.0212572 & Tigre & 0.0179675 \\ \hline
		% 23 & Defensa y Justicia & 0.0206203 & Independiente & 0.0170322 \\ \hline
		% 24 & Quilmes & 0.0168138 & Temperley & 0.0158759 \\ \hline
		% 25 & Godoy Cruz & 0.0166558 & Quilmes & 0.0158634 \\ \hline
		% 26 & Colón & 0.0159257 & Crucero del Norte & 0.0127548 \\ \hline
		% 27 & Olimpo & 0.0135905 & Colón & 0.0106429 \\ \hline
		% 28 & Atlético de Rafaela & 0.0130908 & Olimpo & 0.00977677 \\ \hline
		% 29 & Crucero del Norte & 0.0127524 & Atlético de Rafaela & 0.00877535 \\ \hline
		% 30 & Nueva Chicago & 0.0101696 & Nueva Chicago & 0.00672374 \\ \hline
	\end{tabular}
	%\end{flushright}
	\caption{\footnotesize Primeras 5 posiciones de las fechas 12 y 13 calculadas con GeM y $c = 0.85$.}
	\label{tab:fecha12-13}
\end{table}

A continuación veamos, con un experimento muy simple pero ilustrativo, como el tomar un modelo que ignora los empates frente a otro que les da cierta importancia puede afectar sensiblemente la tabla de posiciones. Para eso veamos que sucede al considerar los resultados de la primera fecha que se muestran en la tabla \ref{tab:fecha1}. Rápidamente, podemos notar que en ambos casos los equipos que comparten el primer lugar son los mismos: los que ganaron su respectivo partido. Sin embargo, en la versión GeM tanto los equipos que empataron como los que perdieron tienen todos la última posición, cosa que no ocurre en el modelo estándar, donde los equipos que empataron están segundos mientras que los que perdieron están últimos. Esto en principio parece un rankeo un poco injusto para los que empataron, que son puestos al mismo nivel de los perdedores.

Vale decir que esta es una situación muy particular: al ser la primer fecha, la matriz de adyacencias no solo será muy esparsa, sino que al menos la mitad de sus filas serán 0 (correspondientes a los equipos que ganaron o empataron)mientras que de la otra mitad a lo sumo tendrán un coeficiente no nulo (correspondientes a los equipos que perdieron). Debido a esto, es que, por ejemplo, todos los equipos que ganaron tienen el mismo puntaje a pesar de que hayan ganado por distintas diferencias (donde la más grande fue de River por 3 goles). Por lo tanto, este debe considerarse como un ejemplo particular donde el dar puntos a los empates genera una situación un poco más justa, pero no debe extrapolarse una conclusión apresurada y afirmar que esto valga para cualquier caso. La experimentación necesaria, así como el análisis que se requiere, para determinar esas conclusiones son de una dificultad no trivial (especialmente considerando que en el caso de GeM el puntaje de un equipo no depende sólo de sus resultados sino también de los de sus adversarios), por lo que no lo abordaremos en más profundidad en este trabajo.

\begin{table}[H]
	\center
	\begin{flushright}
	\begin{tabular}{| m{0.405\textwidth} || m{0.3925\textwidth} |}
		\rowcolor{LightCyan}
		\hline Método GeM & Método Estándar \\ \hline
	\end{tabular}

	\begin{tabular}{| c | c | c || c | c |}
	  	\hline
	  	\rowcolor{LightCyan}
	  	Posición & Equipo & Puntaje & Equipo & Puntaje \\ \hline \hline
		1 & \cellcolor{green!20}River Plate & 0.0450669 & \cellcolor{green!20}Lanús & 3 \\ \hline
		2 & \cellcolor{green!20}Lanús & 0.0450669 & \cellcolor{green!20}Vélez Sarsfield & 3 \\ \hline 
		3 & \cellcolor{green!20}Vélez Sarsfield & 0.0450669 & \cellcolor{green!20}Unión & 3 \\ \hline 
		4 & \cellcolor{green!20}Temperley & 0.0450669 & \cellcolor{green!20}Temperley & 3 \\ \hline 
		5 & \cellcolor{green!20}Rosario Central & 0.0450669 & \cellcolor{green!20}San Lorenzo & 3 \\ \hline 
		6 & \cellcolor{green!20}Independiente & 0.0450669 & \cellcolor{green!20}Rosario Central & 3 \\ \hline 
		7 & \cellcolor{green!20}Estudiantes (LP) & 0.0450669 & \cellcolor{green!20}River Plate & 3 \\ \hline 
		8 & \cellcolor{green!20}Argentinos Juniors & 0.0450669 & \cellcolor{green!20}Independiente & 3 \\ \hline 
		9 & \cellcolor{green!20}Boca Juniors & 0.0450669 & \cellcolor{green!20}Estudiantes (LP) & 3 \\ \hline 
		10 & \cellcolor{green!20}Belgrano & 0.0450669 & \cellcolor{green!20}Defensa y Justicia & 3 \\ \hline 
		11 & \cellcolor{green!20}Defensa y Justicia & 0.0450669 & \cellcolor{green!20}Argentinos Juniors & 3 \\ \hline 
		12 & \cellcolor{green!20}Unión & 0.0450669 & \cellcolor{green!20}Belgrano & 3 \\ \hline 
		13 & \cellcolor{green!20}San Lorenzo & 0.0450669 & \cellcolor{green!20}Boca Juniors & 3 \\ \hline
		14 & \cellcolor{yellow!20}Olimpo & 0.0243606 & \cellcolor{yellow!20}Crucero del Norte & 1 \\ \hline
		15 & \cellcolor{yellow!20}Tigre & 0.0243606 & \cellcolor{yellow!20}Tigre & 1 \\ \hline
		16 & \cellcolor{yellow!20}Arsenal & 0.0243606 & \cellcolor{yellow!20}San Martín (SJ) & 1 \\ \hline
		17 & \cellcolor{yellow!20}Sarmiento & 0.0243606 & \cellcolor{yellow!20}Godoy Cruz & 1 \\ \hline
		18 & \cellcolor{yellow!20}San Martín (SJ) & 0.0243606 & \cellcolor{red!20}Newell's Old Boys & 0 \\ \hline
		19 & \cellcolor{yellow!20}Atlético de Rafaela & 0.0243606 & \cellcolor{red!20}Nueva Chicago & 0 \\ \hline
		20 & \cellcolor{yellow!20}Banfield & 0.0243606 & \cellcolor{red!20}Olimpo & 0 \\ \hline
		21 & \cellcolor{yellow!20}Racing Club & 0.0243606 &\cellcolor{red!20} Quilmes & 0 \\ \hline
		22 & \cellcolor{yellow!20}Quilmes & 0.0243606 &\cellcolor{red!20} Racing Club & 0 \\ \hline
		23 & \cellcolor{yellow!20}Colón & 0.0243606 & \cellcolor{red!20}Aldosivi & 0 \\ \hline
		24 & \cellcolor{yellow!20}Nueva Chicago & 0.0243606 & \cellcolor{red!20}Colón & 0 \\ \hline
		25 & \cellcolor{yellow!20}Newell's Old Boys & 0.0243606 & \cellcolor{red!20}Banfield & 0 \\ \hline
		26 & \cellcolor{yellow!20}Aldosivi & 0.0243606 & \cellcolor{red!20}Huracán & 0 \\ \hline
		27 & \cellcolor{yellow!20}Huracán & 0.0243606 & \cellcolor{red!20}Sarmiento & 0 \\ \hline
		28 & \cellcolor{yellow!20}Godoy Cruz & 0.0243606 & \cellcolor{red!20}Atlético de Rafaela & 0 \\ \hline
		29 & \cellcolor{yellow!20}Gimnasia y Esgrima (LP) & 0.0243606  & \cellcolor{red!20}Arsenal & 0 \\ \hline
		30 & \cellcolor{yellow!20}Crucero del Norte & 0.0243606 &\cellcolor{red!20} Gimnasia y Esgrima (LP) & 0 \\ \hline
	\end{tabular}
	\end{flushright}
	\caption{\footnotesize Ranking correspondiente a la fecha 1 usando el método GeM, para un valor de  $c = 0.85$, y el método estándar. Los sombreados verde, amarillo y rojo señalan la posición del equipo que puede ser 1º, 2º o 3º respectivamente.}
	\label{tab:fecha1}
\end{table}

Finalmente, analicemos el efecto de variar el valor del $c$. Para esto consideremos los resultados expuestos en la tabla \ref{tab:variosc}, que nos muestra el rankeo correspondiente a la fecha 26 usando GeM con dos valores extremos de $c$: $0.1$ y $0.99$. Recordar que el primero es el caso en que la probabilidad de que un equipo pierda contra otro depende mucho menos de los resultados previos y tiende a ser la misma para todos, mientras que el segundo es el opuesto: los resultados previos determinan casi totalmente la probabilidad de que un equipo pierda con otro.

Curiosamente vemos que no hay diferencias grandes entre ambos rankings. De hecho en muchos casos, la posición del equipo no cambió. Aunque en efecto, como habíamos supuesto, el $c$ pequeño hizo que las puntuaciones se homogeneizaran, vemos que en la práctica la ventaja que le da a un equipo ser más goleador que otros y ganarle a equipos importantes sigue siendo lo suficientemente significativa como para que el ranking no presente resultados poco deseables (como que un equipo que rankeaba de la mitad para abajo de la tabla con un $c$ alto, pase a estar en los primeros lugares con uno bajo). De hecho, más bien pareciera que pasa lo contrario. 

Si volvemos a considerar el particular caso de Aldosivi, vemos que al tomar un $c$ más cercano a 0 su posición en la tabla empeoró en cuatro posiciones (respecto de las versiones con $c=0.85$ $c=0.99$). Esto es muy razonable si tenemos en cuenta que el buen rankeo de Aldosivi estaba fuertemente basado en un resultado aislado como fue su amplia victoria sobre Boca, el equipo que mejor $performance$ tuvo en el campeonato. No sólo eso, sino que también le gano por 1-0 a San Lorenzo, otro de los principales equipos. Al reducir la incidencia de los resultados de todos los partidos sobre el cálculo de los puntajes, Aldosivi recibe muchos menos ``votos'' de Boca o San Lorenzo, pues los mismos ahora se reparten más equitativamente entre todos los equipos, y como Aldosivi tuvo pocas victorias en general se ve perjudicado con este cambio.

De la misma forma, pasa algo muy parecido con Vélez (el equipo que más posiciones bajó al considerar el $c$ más chico) que también le ganó a Boca (2-0) y al mismo Aldosivi (2-0), pero en términos generales apenas pudo ganar 6 partidos y perdió 13.

De hecho, la tendencia general es que la subtabla correspondiente a $c=0.1$ tiende a asemejarse mucho más al ranking oficial, que puede verse en la tabla \ref{tab:fecha26}, de que lo hace la subtabla correspondiente a $c=0.99$. Recordar que el método estándar no considera las diferencias de goles, y sí considera la cantidad de victorias, derrotas y empates. Podemos pensar entonces, basándonos en lo que venimos viendo, que un valor más chico de $c$ genera que la cantidad de victorias pese más y al mismo tiempo reduce la importancia de goleadas muy abultadas, lo cual puede ser útil para reducir el impacto de resultados aislados como el de Aldosivi o el de Velez.

En resumen, llegamos a la conclusión de que, contra lo que habíamos supuesto en las hipótesis, la utilización de un $c$ bajo puede dar resultados más sensatos que la de uno muy alto, pues evita que equipos que muestran un bajo rendimiento logren un excelente rankeo sólo por \emph{batacazos} aislados. Por el contrario, el $c$ demasiado alto acrecenta estas diferencias. Particularmente, encontramos que, al menos para este caso, un valor de $c=0.35$ resultado bastante apropiado, basándonos en que, en sentido decreciente, es el primer valor para el cual Aldosivi pasa a estar por debajo de Racing (14 ganados-5 perdidos), que mostró un mejor nivel a lo largo del torneo. No ponemos el ranking obtenido para este valor dado que no aporta nada nuevo a la discución.

\begin{table}[H]
	\center
	\begin{flushright}
	\begin{tabular}{| m{0.405\textwidth} || m{0.416\textwidth} |}
		\rowcolor{LightCyan}
		\hline $c=0.1$ & $c=0.99$ \\ \hline
	\end{tabular}

	\begin{tabular}{| c | c | c || c | c |}
	  	\hline
	  	\rowcolor{LightCyan}
	  	Posición & Equipo & Puntaje & Equipo & Puntaje \\ \hline \hline
		1 & Boca Juniors & 0.0391469 & Boca Juniors & 0.0945905 \\ \hline
		2 & San Lorenzo & 0.0370791 & \cellcolor{green!20}Aldosivi & 0.0744865 \\ \hline
		3 & River Plate & 0.0368742 & River Plate & 0.0675502 \\ \hline
		4 & Racing Club & 0.0357606 & San Lorenzo & 0.0656075 \\ \hline
		5 & Rosario Central & 0.0353159 & Rosario Central & 0.0505593 \\ \hline
		6 & \cellcolor{green!20}Aldosivi & 0.0347251 & Racing Club & 0.049052 \\ \hline
		7 & Quilmes & 0.0344085 & San Martín (SJ) & 0.0468815 \\ \hline
		8 & Independiente & 0.0341835 & Quilmes & 0.0433725 \\ \hline
		9 & Belgrano & 0.034173 & \cellcolor{blue!20}Vélez Sarsfield & 0.0410949 \\ \hline
		10 & San Martín (SJ) & 0.033846 & Newell's Old Boys & 0.039274 \\ \hline
		11 & Newell's Old Boys & 0.0337048 & Independiente & 0.0355528 \\ \hline
		12 & Gimnasia y Esgrima (LP) & 0.033465 & Belgrano & 0.0354938 \\ \hline
		13 & Lanús & 0.0334376 & Gimnasia y Esgrima (LP) & 0.0337113 \\ \hline
		14 & Banfield & 0.0332383  & Banfield & 0.0304023 \\ \hline
		15 & Tigre & 0.0331063 & Estudiantes (LP) & 0.0290798 \\ \hline
		16 & Estudiantes (LP) & 0.0326988 & Unión & 0.0283743 \\ \hline
		17 & Unión & 0.0325988 & Tigre & 0.0260657 \\ \hline
		18 & Sarmiento & 0.0325042 & Sarmiento & 0.0245763 \\ \hline
		19 & \cellcolor{blue!20}Vélez Sarsfield & 0.0324059 & Defensa y Justicia & 0.023483 \\ \hline
		20 & Arsenal & 0.0322979 & Huracán & 0.0229007 \\ \hline
		21 & Huracán & 0.0322357 & Lanús & 0.0220692 \\ \hline
		22 & Defensa y Justicia & 0.0318777 & Olimpo & 0.0215218 \\ \hline
		23 & Olimpo & 0.0318753 & Arsenal & 0.0179464 \\ \hline
		24 & Argentinos Juniors & 0.0317518 & Godoy Cruz & 0.0142913 \\ \hline
		25 & Godoy Cruz & 0.0317468 & Crucero del Norte & 0.012915 \\ \hline
		26 & Temperley & 0.0315835 & Temperley & 0.0124947 \\ \hline
		27 & Crucero del Norte & 0.0313297 & Argentinos Juniors & 0.0115357 \\ \hline
		28 & Nueva Chicago & 0.0310144 & Nueva Chicago & 0.0114375 \\ \hline
		29 & Atlético de Rafaela & 0.0308366 & Atlético de Rafaela & 0.00759916 \\ \hline
		30 & Colón & 0.0307777 & Colón & 0.00608033 \\ \hline
	\end{tabular}
	\end{flushright}
	\caption{\footnotesize Ranking correspondiente a la fecha 26 usando el método GeM para dos valores de $c$ distintos: $0.1$ y $0.99$}
	\label{tab:variosc}
\end{table}
